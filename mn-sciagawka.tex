\documentclass[10pt,a4paper,twocolumn]{article}

\usepackage{multicol}
\usepackage{amsbsy, amssymb, latexsym, amsmath, braket}
\usepackage[tiny]{titlesec}
\usepackage[hmargin=0.5cm,vmargin=0.7cm]{geometry}
\usepackage[utf8x]{inputenc}
\usepackage{polski}
\usepackage{scalefnt}
\usepackage[yyyymmdd,hhmmss]{datetime}
\usepackage{commath}
\usepackage[b]{esvect}  % lepsze strzałki nad wektorami
\usepackage{nicematrix} % łatwe macierze blokowe

% Potrzebne do algorytmu Euklidesa.
\usepackage{tikz}
\usetikzlibrary{tikzmark}

\newcommand{\angles}[1]{\left\langle #1 \right\rangle}


\newcommand{\entry}{$\bullet$\hspace{0.15em}}
\newcommand{\subentry}{$\circledcirc$\hspace{0.15em}}
% https://tex.stackexchange.com/a/7045/80219
\newcommand{\textsubentry}[1]{\tikz[baseline=(char.base)]{
            \node[shape=circle,draw,inner sep=1pt] (char) {#1};\hspace{0.15em}}}


\titlespacing{\section}{0pt}{0pt}{0pt}
\titlespacing{\subsection}{0pt}{0pt}{0pt}
\titlespacing{\subsubsection}{0pt}{0pt}{0pt}

% Wyłącz numerowanie stron.
\pagenumbering{gobble}

\setlength{\parindent}{0pt}
% Odległość pomiędzy liniami. Zmniejsz, jeżeli brakuje miejsca.
\setlength{\parskip}{0.5ex}

\title{Karta wzorów z matematyki dyskretnej}

\begin{document}
% Rozmiar czcionki.
\scalefont{.48}

\text{\tiny{
    Wersja z \today\ o \currenttime\ (\pdfmdfivesum file{./mn-sciagawka.tex})
}}

% Układy równań liniowych
\section{Układy równań liniowych}
\subsection{Podstawowe pojęcia}
\entry
$Ax = b$ ma rozwiązanie gdy: $det(A) \neq 0$, $b = 0$ to $x = 0$, $ker(A) = \{0\}$, $A$ nie ma zerowej wartości własnej, $A$ jest odwracalna.

\entry
Macierz z diagonalą rozwiązuje się trywialnie: $x_i = \frac{b_i}{a_{ii}}$. Koszt $O(n)$.

\entry
FLOP - FLoating point OPeration - dodawanie, mnożenie, dzielenie, odejmowanie, sqrt to jeden flop.

\entry
Macierz trójkątna dolna, układ równań rozwiązany w przód: $x_i = \frac{b_i - \Sigma_{j=1}^{i-1} a_{ji}x_j}{a_{ii}}$. Koszt $O(n^2)$.

\entry
Macierz trójkątna górna rozwiązanie analogiczne (w tył), koszt $O(n^2)$.

\entry
Macierz ortogonalna $Q$ spełnia $Q^TQ = I$, $Q^{-1} = Q^T$.

\entry
Gdy $A$ ortogonalna, to $Ax = b$ ma rozwiązanie $x = A^Tb$. Koszt $O(n^2)$.

\subsection{Rozkład LU}
\entry
$PA = LU$ (z wyborem) lub $A = LU$ (bez wyboru), $P$ - permutacja, $L$ - trójkątna dolna, $U$ - trójkątna górna.

\entry
Standardowy algorytm rozkładu LU z bez wyboru elementu głównego: $A = LU$.\\
$  PA =
    \begin{bNiceArray}{c|c}
      a & \vv{w}^T
      \\
      \hline
      \vv{v} & A'
    \end{bNiceArray} =
    \begin{bNiceArray}{c|c}
      1 & \vv{0}^T
      \\
      \hline
      \vv{w}/a & I
    \end{bNiceArray} \cdot
    \begin{bNiceArray}{c|c}
      a & \vv{w}^T
      \\
      \hline
      \vv{0} & {A'-\vv{v}\vv{w}^T/a}
    \end{bNiceArray}
$\\
Powtórz dla $U_{22}$.
Działa gdy wszystkie minory główne macierzy $A$ są niezerowe.

\entry
GEPP - Gauss Elimination with Partial Pivoting - GE z częściowym wyborem elementu głównego.\\
$PA = LU$, $P$ - permutacja, $L$ - trójkątna dolna, $U$ - trójkątna górna.\\
$PAx = LUx = Pb$, $\tilde{b} = Pb$, $Lx = \tilde{b}$, $Uy = y$. Koszt $O(n^3)$.


\section{Arytmetyka zmiennopozycyjna}

\entry
$6.63 \cdot 10^{-34}$ - $6.63$ - mantysa, $-34$ - cecha, $10$ - podstawa

\entry
$x = (-1)^s \cdot m \cdot b^e$ - $s$ - znak, $m = (f_0.f_1f_2...f_{p-1)_2}$ - mantysa, $b$ - podstawa, $e$ - cecha

\entry
W liczbach maszynowych $\beta = 2$

\entry
$(1 - e_{max} \leq e \leq e_{max})$

\entry
Liczby maszynowe są normalizowane $f_0 = 1$ i nie jest zapisywane.

\entry
W 6-bitowej arytmetyce zmiennopozycyjnej liczby subnormalne zachodzą dla $e=-2$ i $f_0 = 0$.

\entry
Metody aproksymacji liczb maszynowych:\\
RN - do najbliższej (domyślnie)\\
RD - w dół, tzn w stone -$\infty$\\
RU - w górę, tzn w stronę $\infty$\\
RZ - do zera $RZ(x) = RD(x)$ gdy $x \geq 0$,$RU(x)$ gdy $x \leq 0$

\entry
Jeśli $|x| \in [realmin, realmax]$ to $\frac{|x - RN(x)|}{|x|} \leq \frac{1}{2^p} =v$

\entry 
$fl(x) = x (1 \cdot \epsilon)$, gdy $|\epsilon| \leq v$.


\end{document}