\documentclass[10pt,a4paper,twocolumn]{article}

\usepackage{multicol}
\usepackage{amsbsy, amssymb, latexsym, amsmath, braket}
\usepackage{amssymb}
\usepackage[tiny]{titlesec}
\usepackage[hmargin=0.5cm,vmargin=0.7cm]{geometry}
\usepackage[utf8x]{inputenc}
\usepackage{polski}
\usepackage{scalefnt}
\usepackage[yyyymmdd,hhmmss]{datetime}
\usepackage{commath}

% Potrzebne do algorytmu Euklidesa.
\usepackage{tikz}
\usetikzlibrary{tikzmark}

\newcommand{\angles}[1]{\left\langle #1 \right\rangle}


\newcommand{\entry}{$\bullet$\hspace{0.15em}}
\newcommand{\subentry}{$\circledcirc$\hspace{0.15em}}
% https://tex.stackexchange.com/a/7045/80219
\newcommand{\textsubentry}[1]{\tikz[baseline=(char.base)]{
            \node[shape=circle,draw,inner sep=1pt] (char) {#1};\hspace{0.15em}}}


\titlespacing{\section}{0pt}{0pt}{0pt}
\titlespacing{\subsection}{0pt}{0pt}{0pt}
\titlespacing{\subsubsection}{0pt}{0pt}{0pt}

% Wyłącz numerowanie stron.
\pagenumbering{gobble}

\setlength{\parindent}{0pt}
% Odległość pomiędzy liniami. Zmniejsz, jeżeli brakuje miejsca.
\setlength{\parskip}{0.5ex}

\title{Karta wzorów z matematyki dyskretnej}

\begin{document}
% Rozmiar czcionki.
\scalefont{.8}

\text{\tiny{
    Wersja z \today\ o \currenttime\ (\pdfmdfivesum file{./mn-sciagawka.tex})
}}

\section{Metody iteracyjne}

\entry
Ciąg $x_{k+1} = Bx_k + c$ jest zbieżny do $x^*$ dla każdego $x_0$ wtedy i tylko wtedy, gdy $\rho(B) < 1$. gdzie $\rho(B)$ to promień spektralny macierzy $B$ ($\rho(B) = max\{|\lambda| : \lambda \text{ jest w. wł. B}\}$)

\entry
Jeśli $\norm{B} < 1$ to ciąg $x_{k+1} = Bx_k + c$ jest zbieżny do $x^*$ dla każdego $x_0$. oraz $\norm{x^* - x_k} \leq \norm{B} \cdot \norm{x_k - x^*}$

\entry 
Niech $A = M - Z$ oraz $A,M$ nieosobliwe. $Ax^* = b$. Jeśli $\rho(M^{-1}Z) < 1$ to metoda iteracyjna $Mx_{k+1} = Zx_k + b$ jest zbieżna do $x^*$ dla każdego $x_0$.\\
Jeśli dodatkowo $\gamma = \norm{M^{-1}Z}_\infty < 1$ to $\norm{x^* - x_k} \leq \gamma \cdot \norm{x_k - x^*}$

\entry
Niech $A = L + D + U$, $D = diag(A)$, $L$ (odp. U) - dolna (odp. górna) trójkątna z zerową diagonalą.\\
$x_{k+1} = x_k + M^{-1}(b-Ax_k)$\\
Metoda Jacobiego: $M = D$, Metoda Gaussa-Seidela: $M = D + L$, Metoda SOR: $M = 1/\omega D + L$.

\entry
Jeśli $A$ jest diagonalnie dominująca to m. Jacobiego jest zbieżna do $x^*$ dla każdego $x_0$.

\entry 
$x_{k+1} = x_{k} + \delta_k$. Jak wybrać $\delta_k$?. Idealna poprawka $\delta_k^*$:\\
$A\delta_k^* = r_k \rightarrow x_{k+1} = x^*$\\
Wyznaczamy idealną poprawkę $\delta_k^* = V_k a_k$, gdzie $V_k, U_k \in \mathbb{R}^{N \times r}$ max rzędu t. że $a_k \in \mathbb{R}^r$ spełnia:\\
$U_k^TAV_k a_k = U_k^Tr_k$\\

\end{document}