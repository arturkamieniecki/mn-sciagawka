\documentclass[10pt,a4paper,twocolumn]{article}

\usepackage{multicol}
\usepackage{amsbsy, amssymb, latexsym, amsmath, braket}
\usepackage[tiny]{titlesec}
\usepackage[hmargin=0.5cm,vmargin=0.7cm]{geometry}
\usepackage[utf8x]{inputenc}
\usepackage{polski}
\usepackage{scalefnt}
\usepackage[yyyymmdd,hhmmss]{datetime}
\usepackage{commath}

% Potrzebne do algorytmu Euklidesa.
\usepackage{tikz}
\usetikzlibrary{tikzmark}

\newcommand{\angles}[1]{\left\langle #1 \right\rangle}


\newcommand{\entry}{$\bullet$\hspace{0.15em}}
\newcommand{\subentry}{$\circledcirc$\hspace{0.15em}}
% https://tex.stackexchange.com/a/7045/80219
\newcommand{\textsubentry}[1]{\tikz[baseline=(char.base)]{
            \node[shape=circle,draw,inner sep=1pt] (char) {#1};\hspace{0.15em}}}


\titlespacing{\section}{0pt}{0pt}{0pt}
\titlespacing{\subsection}{0pt}{0pt}{0pt}
\titlespacing{\subsubsection}{0pt}{0pt}{0pt}

% Wyłącz numerowanie stron.
\pagenumbering{gobble}

\setlength{\parindent}{0pt}
% Odległość pomiędzy liniami. Zmniejsz, jeżeli brakuje miejsca.
\setlength{\parskip}{0.5ex}

\title{Karta wzorów z matematyki dyskretnej}

\begin{document}
% Rozmiar czcionki.
\scalefont{.48}

\text{\tiny{
    Wersja z \today\ o \currenttime\ (\pdfmdfivesum file{./mn-sciagawka.tex})
}}

\section{Uwarunkowanie zad. i numeryczna poprawnosc}

\entry
$\norm{\tilde{x}-x}$ - błąd bezwzględny, $\norm{\tilde{x}-x}/\norm{x}$ - błąd względny ($x \neq 0$)

\entry
Wskaźnik uwarunkowania (bezwzględny) na poziomie $\epsilon$\\
$cond_{abs}(P,x,\epsilon) = \sup_{\norm{\Delta} \leq \epsilon} \frac{\norm{P(x+\Delta) - P(x)}}{\norm{\Delta}}$\\
$\norm{P(\tilde{x}) - P(x)} \leq cond_{abs}(P,x,\epsilon) \cdot \norm{\tilde{x}-x}$, dla $\norm{\tilde{x}-x} \leq \epsilon$

\entry
Idealizacja punktowy wskaźnik uwarunkowania\\
$cond_{abs}(P,x) = \lim_{\epsilon \to 0} cond_{abs}(P,x,\epsilon)$

\entry
Gdy P różniczkowalna to $cond_{abs}(P,x) = \norm{P'(x)}$

\entry
Wrażliwość dla błędu względnego\\
$cond_{rel}(P,x,\epsilon) = \sup_{\norm{\Delta} \leq \epsilon} \frac{\norm{P(x+\Delta) - P(x)}}{\norm{P(x)}} / \frac{\norm{\Delta}}{\norm{x}} = cond_{abs}(P,x,\epsilon\norm{x})\frac{\norm{x}}{\norm{P(x)}}$\\
Punktowo:\\
$cond_{rel}(P,x) = \lim_{\epsilon \to 0} cond_{rel}(P,x,\epsilon)$\\
Gdy P jest różniczkowalna to $cond_{rel}(P,x) = \frac{\norm{P'(x)}}{\norm{P(x)}}$

\entry
Zadanie dobrze uwarunkowanie: $cond(P,x)$ nieduże\\
Zadanie źle uwarunkowane: $cond(P,x)$ bardzo duże 

\entry
Normy wektorowe\\
$\norm{x}_1 = \sum_{i=1}^n |x_i|$, $\norm{x}_2 = \sqrt{\sum_{i=1}^n |x_i|^2}$, $\norm{x}_p = (\sum_{i=1}^n |x_i|^p)^{1/p}$, $\norm{x}_\infty = \max_{1 \leq i \leq n} |x_i|$

\entry
$\norm{x}_\infty \leq \norm{x}_1 \leq N \norm{x}_\infty$, $\norm{x}_\infty \leq \norm{x}_2 \leq \sqrt{n} \norm{x}_\infty$, $\norm{x}_2 \leq \norm{x}_1 \leq \sqrt{n} \norm{x}_2$

\entry
$\norm{A} = max_{\norm{x} \neq 0} \norm{Ax} / \norm{x} = max_{\norm{x}=1}\norm{Ax} = max_{\norm{x} \leq 1}\norm{Ax}$

\entry
$\norm{Ax} \leq \norm{A} \norm{x}$, $\norm{AB} \leq \norm{A} \norm{B}$, $\norm{I} = 1$, $\norm{A}_1 = max_j \Sigma_i |a_{ij}|$, $\norm{A}_\infty = max_i \Sigma_j |a_{ij}|$, $\norm{A}_2 =max\{\sqrt{\mu}: \mu \text{ jest w. wł. }A^TA\}$

\entry 
$cond(A) = \norm{A} \norm{A^{-1}}$

\entry
Ay = b. Jeśli $\epsilon cond(A) \leq 1/2$ to $\norm{\tilde{y}-y}/\norm{y} \leq 4cond(A)\cdot\epsilon$

\entry
Jeśli $\norm{\Delta} < 1$ to $I + \Delta$ nieosobliwa i $1/(1+\norm{\Delta}) \leq \norm{(I+\Delta)^{-1}} \leq 1/(1-\norm{\Delta})$

\entry
Algorytm poprawnie numeryczny - dla każdego $x\in X$ wynik algorytmu A zrealizowanego w fl $fl(A(fl(x)))$ jest dokładnym rozwiązaniem zadania dla danych x zaburzonych na poziomie błędu reprezentacji.

\entry 
Algorytm NP daje wynik, którego błąd można oszacować na podstawie własności zadania obliczeniowego: \\
$\norm{\tilde{y} - y} / \norm{y} = \norm{P(\tilde{x}) - P(x)}/\norm{P(x)} \lesssim cond_{rel}(P, x)\norm{\tilde{x}-x}/\norm{x} \leq K \cdot cond_{rel}(P,x) \cdot v$
\section{Arytmetyka zmiennopozycyjna}

\entry
$6.63 \cdot 10^{-34}$ - $6.63$ - mantysa, $-34$ - cecha, $10$ - podstawa

\entry
$x = (-1)^s \cdot m \cdot b^e$ - $s$ - znak, $m = (f_0.f_1f_2...f_{p-1)_2}$ - mantysa, $b$ - podstawa, $e$ - cecha

\entry
W liczbach maszynowych $\beta = 2$

\entry
$(1 - e_{max} \leq e \leq e_{max})$

\entry
Liczby maszynowe są normalizowane $f_0 = 1$ i nie jest zapisywane.

\entry
W 6-bitowej arytmetyce zmiennopozycyjnej liczby subnormalne zachodzą dla $e=-2$ i $f_0 = 0$.

\entry
Metody aproksymacji liczb maszynowych:\\
RN - do najbliższej (domyślnie)\\
RD - w dół, tzn w stone -$\infty$\\
RU - w górę, tzn w stronę $\infty$\\
RZ - do zera $RZ(x) = RD(x)$ gdy $x \geq 0$,$RU(x)$ gdy $x \leq 0$

\entry
Jeśli $|x| \in [realmin, realmax]$ to $\frac{|x - RN(x)|}{|x|} \leq \frac{1}{2^p} =v$

\entry 
$fl(x) = x (1 \cdot \epsilon)$, gdy $|\epsilon| \leq v$.

\end{document}
