\documentclass[10pt,a4paper,twocolumn]{article}

\usepackage{multicol}
\usepackage{amsbsy, amssymb, latexsym, amsmath, braket}
\usepackage[tiny]{titlesec}
\usepackage[hmargin=0.5cm,vmargin=0.7cm]{geometry}
\usepackage[utf8x]{inputenc}
\usepackage{polski}
\usepackage{scalefnt}
\usepackage[yyyymmdd,hhmmss]{datetime}
\usepackage{commath}
\usepackage[b]{esvect}  % lepsze strzałki nad wektorami
\usepackage{nicematrix} % łatwe macierze blokowe


% Potrzebne do algorytmu Euklidesa.
\usepackage{tikz}
\usetikzlibrary{tikzmark}

\newcommand{\angles}[1]{\left\langle #1 \right\rangle}


\newcommand{\entry}{$\bullet$\hspace{0.15em}}
\newcommand{\subentry}{$\circledcirc$\hspace{0.15em}}
% https://tex.stackexchange.com/a/7045/80219
\newcommand{\textsubentry}[1]{\tikz[baseline=(char.base)]{
            \node[shape=circle,draw,inner sep=1pt] (char) {#1};\hspace{0.15em}}}


\titlespacing{\section}{0pt}{0pt}{0pt}
\titlespacing{\subsection}{0pt}{0pt}{0pt}
\titlespacing{\subsubsection}{0pt}{0pt}{0pt}

% Wyłącz numerowanie stron.
\pagenumbering{gobble}

\setlength{\parindent}{0pt}
% Odległość pomiędzy liniami. Zmniejsz, jeżeli brakuje miejsca.
\setlength{\parskip}{0.5ex}

\title{Karta wzorów z matematyki dyskretnej}

\begin{document}
% Rozmiar czcionki.
\scalefont{.48}

\text{\tiny{
    Wersja z \today\ o \currenttime\ (\pdfmdfivesum file{./mn-sciagawka.tex})
}}

% Układy równań liniowych
\section{Układy równań liniowych}
\subsection{Podstawowe pojęcia}
\entry
$Ax = b$ ma rozwiązanie gdy: $det(A) \neq 0$, $b = 0$ to $x = 0$, $ker(A) = \{0\}$, $A$ nie ma zerowej wartości własnej, $A$ jest odwracalna.

\entry
Macierz z diagonalą rozwiązuje się trywialnie: $x_i = \frac{b_i}{a_{ii}}$. Koszt $O(n)$.

\entry
FLOP - FLoating point OPeration - dodawanie, mnożenie, dzielenie, odejmowanie, sqrt to jeden flop.

\entry
Macierz trójkątna dolna, układ równań rozwiązany w przód: $x_i = \frac{b_i - \Sigma_{j=1}^{i-1} a_{ji}x_j}{a_{ii}}$. Koszt $O(n^2)$.

\entry
Macierz trójkątna górna rozwiązanie analogiczne (w tył), koszt $O(n^2)$.

\entry
Macierz ortogonalna $Q$ spełnia $Q^TQ = I$, $Q^{-1} = Q^T$.

\entry
Gdy $A$ ortogonalna, to $Ax = b$ ma rozwiązanie $x = A^Tb$. Koszt $O(n^2)$.

\subsection{Rozkład LU}
\entry
$PA = LU$ (z wyborem) lub $A = LU$ (bez wyboru), $P$ - permutacja, $L$ - trójkątna dolna, $U$ - trójkątna górna.

\entry
Standardowy algorytm rozkładu LU z bez wyboru elementu głównego: $A = LU$.\\
$  PA =
    \begin{bNiceArray}{c|c}
      a & \vv{w}^T
      \\
      \hline
      \vv{v} & A'
    \end{bNiceArray} =
    \begin{bNiceArray}{c|c}
      1 & \vv{0}^T
      \\
      \hline
      \vv{w}/a & I
    \end{bNiceArray} \cdot
    \begin{bNiceArray}{c|c}
      a & \vv{w}^T
      \\
      \hline
      \vv{0} & {A'-\vv{v}\vv{w}^T/a}
    \end{bNiceArray}
$\\
Powtórz dla $U_{22}$.
Działa gdy wszystkie minory główne macierzy $A$ są niezerowe.

\entry
GEPP - Gauss Elimination with Partial Pivoting - GE z częściowym wyborem elementu głównego.\\
$PA = LU$, $P$ - permutacja, $L$ - trójkątna dolna, $U$ - trójkątna górna.\\
$PAx = LUx = Pb$, $\tilde{b} = Pb$, $Lx = \tilde{b}$, $Uy = y$. Koszt $O(n^3)$.


\section{Arytmetyka zmiennopozycyjna}

\entry
$6.63 \cdot 10^{-34}$ - $6.63$ - mantysa, $-34$ - cecha, $10$ - podstawa

\entry
$x = (-1)^s \cdot m \cdot b^e$ - $s$ - znak, $m = (f_0.f_1f_2...f_{p-1)_2}$ - mantysa, $b$ - podstawa, $e$ - cecha

\entry
W liczbach maszynowych $\beta = 2$

\entry
$(1 - e_{max} \leq e \leq e_{max})$

\entry
Liczby maszynowe są normalizowane $f_0 = 1$ i nie jest zapisywane.

\entry
W 6-bitowej arytmetyce zmiennopozycyjnej liczby subnormalne zachodzą dla $e=-2$ i $f_0 = 0$.

\entry
Metody aproksymacji liczb maszynowych:\\
RN - do najbliższej (domyślnie)\\
RD - w dół, tzn w stone -$\infty$\\
RU - w górę, tzn w stronę $\infty$\\
RZ - do zera $RZ(x) = RD(x)$ gdy $x \geq 0$,$RU(x)$ gdy $x \leq 0$

\entry
Jeśli $|x| \in [realmin, realmax]$ to $\frac{|x - RN(x)|}{|x|} \leq \frac{1}{2^p} =v$

\entry 
$fl(x) = x (1 \cdot \epsilon)$, gdy $|\epsilon| \leq v$.

\section{Uwarunkowanie zad. i numeryczna poprawnosc}

\entry
$\norm{\tilde{x}-x}$ - błąd bezwzględny, $\norm{\tilde{x}-x}/\norm{x}$ - błąd względny ($x \neq 0$)

\entry
Wskaźnik uwarunkowania (bezwzględny) na poziomie $\epsilon$\\
$cond_{abs}(P,x,\epsilon) = \sup_{\norm{\Delta} \leq \epsilon} \frac{\norm{P(x+\Delta) - P(x)}}{\norm{\Delta}}$\\
$\norm{P(\tilde{x}) - P(x)} \leq cond_{abs}(P,x,\epsilon) \cdot \norm{\tilde{x}-x}$, dla $\norm{\tilde{x}-x} \leq \epsilon$

\entry
Idealizacja punktowy wskaźnik uwarunkowania\\
$cond_{abs}(P,x) = \lim_{\epsilon \to 0} cond_{abs}(P,x,\epsilon)$

\entry
Gdy P różniczkowalna to $cond_{abs}(P,x) = \norm{P'(x)}$

\entry
Wrażliwość dla błędu względnego\\
$cond_{rel}(P,x,\epsilon) = \sup_{\norm{\Delta} \leq \epsilon} \frac{\norm{P(x+\Delta) - P(x)}}{\norm{P(x)}} / \frac{\norm{\Delta}}{\norm{x}} = cond_{abs}(P,x,\epsilon\norm{x})\frac{\norm{x}}{\norm{P(x)}}$\\
Punktowo:\\
$cond_{rel}(P,x) = \lim_{\epsilon \to 0} cond_{rel}(P,x,\epsilon)$\\
Gdy P jest różniczkowalna to $cond_{rel}(P,x) = \frac{\norm{P'(x)}}{\norm{P(x)}}$

\entry
Zadanie dobrze uwarunkowanie: $cond(P,x)$ nieduże\\
Zadanie źle uwarunkowane: $cond(P,x)$ bardzo duże 

\entry
Normy wektorowe\\
$\norm{x}_1 = \sum_{i=1}^n |x_i|$, $\norm{x}_2 = \sqrt{\sum_{i=1}^n |x_i|^2}$, $\norm{x}_p = (\sum_{i=1}^n |x_i|^p)^{1/p}$, $\norm{x}_\infty = \max_{1 \leq i \leq n} |x_i|$

\entry
$\norm{x}_\infty \leq \norm{x}_1 \leq N \norm{x}_\infty$, $\norm{x}_\infty \leq \norm{x}_2 \leq \sqrt{n} \norm{x}_\infty$, $\norm{x}_2 \leq \norm{x}_1 \leq \sqrt{n} \norm{x}_2$

\entry
$\norm{A} = max_{\norm{x} \neq 0} \norm{Ax} / \norm{x} = max_{\norm{x}=1}\norm{Ax} = max_{\norm{x} \leq 1}\norm{Ax}$

\entry
$\norm{Ax} \leq \norm{A} \norm{x}$, $\norm{AB} \leq \norm{A} \norm{B}$, $\norm{I} = 1$, $\norm{A}_1 = max_j \Sigma_i |a_{ij}|$, $\norm{A}_\infty = max_i \Sigma_j |a_{ij}|$, $\norm{A}_2 =max\{\sqrt{\mu}: \mu \text{ jest w. wł. }A^TA\}$

\entry 
$cond(A) = \norm{A} \norm{A^{-1}}$

\entry
Ay = b. Jeśli $\epsilon cond(A) \leq 1/2$ to $\norm{\tilde{y}-y}/\norm{y} \leq 4cond(A)\cdot\epsilon$

\entry
Jeśli $\norm{\Delta} < 1$ to $I + \Delta$ nieosobliwa i $1/(1+\norm{\Delta}) \leq \norm{(I+\Delta)^{-1}} \leq 1/(1-\norm{\Delta})$

\entry
Algorytm poprawnie numeryczny - dla każdego $x\in X$ wynik algorytmu A zrealizowanego w fl $fl(A(fl(x)))$ jest dokładnym rozwiązaniem zadania dla danych x zaburzonych na poziomie błędu reprezentacji.

\entry 
Algorytm NP daje wynik, którego błąd można oszacować na podstawie własności zadania obliczeniowego: \\
$\norm{\tilde{y} - y} / \norm{y} = \norm{P(\tilde{x}) - P(x)}/\norm{P(x)} \lesssim cond_{rel}(P, x)\norm{\tilde{x}-x}/\norm{x} \leq K \cdot cond_{rel}(P,x) \cdot v$

\section{LZNK}
$A \in \mathbb{R}^{m \times n},m \ge n, \text{rank}(A)=n, \vv{b} \in \mathbb{R}^m$, znaleźć $\vv{x} \in \mathbb{R}^n$ taki, że $\lVert \vv{b} - A\vv{x} \rVert_2 \to min$.\\
$\lVert \vv{b} - A\vv{x} \rVert \le \lVert \vv{b} - A\vv{y} \rVert \forall y$.\\
\entry
$QR$ ---
$
A = QR = 
\begin{bmatrix}
  \hat{Q} & \tilde{Q}
\end{bmatrix}
\begin{bmatrix}
  \hat{R}
  \\
  0
\end{bmatrix}
\implies$\\
$
\lVert \vv{b} - A\vv{x} \rVert_2^2 =
\lVert \hat{Q}^T \vv{b} - \hat{R} \vv{x} \rVert_2^2 + \lVert \tilde{Q}^T \vv{b} \rVert_2^2 = \min \iff
\hat{R}\vv{x} = \hat{Q}\vv{b}
$

\entry
$A^TA$ ---
mała, symetryczna, dodatnio określona
$A^TA \vv{x} = A^T \vv{b}$

\entry
Algorytm:
oblicz $B = A^TA \rightarrow$
wyznacz Cholesky'ego $B = LL^T \rightarrow$
rozwiąż $LL^T\vv{x} = A^T \vv{b}$

\entry
$SVD$ ---
$A \in \mathbb{R}^{m \times n}, m \ge n$, istnieje
$A = U \Sigma V^T$, że\\
$
U \in \mathbb{R}^{m \times n}: U^TU = I,
V \in \mathbb{R}^{n \times n}: V^TV = I,
\Sigma \in \mathbb{R}^{n \times n}: \Sigma = \text{diag}\left(\sigma_1, .., \sigma_n\right),
\sigma_1 \ge .. \ge \sigma_n \ge 0
$\\
$
\lambda(A^TA) = (\sigma_1^2, .., \sigma_n^2),
\sigma_1 \ge .. \ge \sigma_r > \sigma_{r+1}=..=\sigma_n = 0 \implies \text{rank}{A} = r
$\\
$ A = \sum_{i=1}^n \sigma_i \vv{u_i} \vv{v_i}^T \text{ gdzie }\vv{u_i},\vv{v_i} \text{ --- kolumny } U,V $\\
$ \lVert \vv{b} - A \vv{x} \rVert_2^2 = \sum_{i=1}^r \left( \sigma_i y_i - \vv{u_i}^T \vv{b} \right)^2 + \sum_{i=r+1}^m \left( \vv{u_i}^T \vv{b} \right)^2 \rightarrow
\min \text{, gdy } y_i = \frac{\vv{u_i}^T \vv{b}}{\sigma_i} $,\\
To oznacza, że rozwiązanie LZNK jest jednoznacznie określone
$\vv{x^*} = \sum_{i=1}^r \frac{\vv{u_i}^T \vv{b}}{\sigma_i}\vv{v_i}$




\end{document}
