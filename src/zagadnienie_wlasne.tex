\section{Zagadnienie własne}
\entry
Znaleźć $\vv{v} \in \mathbb{C}^n$ oraz $\lambda \in \mathbb{C}$ t. że
$A\vv{v} = \lambda\vv{v} \land \norm{\vv{v}} = 1$.

\entry
Każda symetryczna macierz $A\in\mathbb{R}^{n\times n}$ ma rozkład\\
$A = Q \Lambda Q^T$, $\Lambda =
\begin{bmatrix}
  \lambda_1\\
  & \ddots \\
  && \lambda_n
\end{bmatrix}$,\\
$\lambda_i \in \mathbb{R}$ --- wartości własne $A$, kolumny $Q$ --- wektory własne $A$.

\entry
Metoda potęgowa --- zakładamy  ze $|\lambda_1| > |\lambda_2| \ge \hdots \ge |\lambda_n|$,\\
losuj $\vv{x_0}$,\quad $\vv{x_{k+1}} = A\vv{x_k} / \norm{A \vv{x_k}}$,
znajduje $\lambda_{\text{max}}$.

\entry
Wyznaczanie wartości własnej na podstawie wektora własnego.
$\norm{A\vv{v} - \lambda\vv{v}} \rightarrow \min$,
$\lambda = \frac{\vv{x}^H A\vv{v}}{\vv{x}^H\vv{x}}$

\entry
Odwrotna metoda potęgowa --- zakładamy  ze $|\lambda_1| \ge \hdots \ge |\lambda_n| > 0$,
wyznacz $PA = LU$, losujemy $\vv{x_0}$,
rozwiąż $LU\vv{x_{k+1}} = P\vv{x_k}$, przeskaluj $\vv{x_{k+1}} = \vv{x_{k+1}}/\norm{\vv{x_{k+1}}}$,
znajduje $1 / \lambda_{\text{min}}$.

\entry
Odwrotna metoda potęgowa z parametrem --- to samo, co odwrotna ale dla $(A - \mu)^{-1}$,
znajduje $\lambda$ najbliższe $\mu$.

\entry
$\forall_{i} \lambda_i(A) \le \norm{A}$

\entry
Koła Greszgorina --- $\forall_{\lambda} \exists_{i} |\lambda - a_{ii}| \le \sum_{j\neq i}|a_{ij}|$

\entry
\begin{tabular}{|
    >{\centering}p{2cm}|
    >{\centering}p{2cm}|
    >{\centering}p{1.5cm}|
    >{\centering\arraybackslash}p{2cm}
  |}
  \hline
  Macierz & wart. wł. & wekt. wł. & zastrz.\\
  \hline
  $A$ & $\lambda$ & $\vv{v}$ & --- \\
  \hline
  $A - \mu I$ & $\lambda - \mu$ & $\vv{v}$ & --- \\
  \hline
  $A^{-1}$ & $1/\lambda$ & $\vv{v}$ & $A$ nieos. \\
  \hline
  $(A - \mu I)^{-1}$ & $1/(\lambda - \mu)$ & $\vv{v}$ & $A-\mu$ nieos. \\
  \hline
\end{tabular}
